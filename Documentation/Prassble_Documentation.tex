\documentclass[11pt, a4paper]{book}
%% The LaTeX project Prassble
%% preamble.tex: the preamble of Prassble
%%
%% -------------------------------------------------------------------------------------------
%% Copyright (c) 2025 by Grass (GrassGlass) <shaohong00002 at gmail dot com>
%% -------------------------------------------------------------------------------------------
%%
%% This work may be distributed and/or modified under the
%% conditions of the LaTeX Project Public License, either version 1.3
%% of this license or (at your option) any later version.
%% The latest version of this license is in
%%   http://www.latex-project.org/lppl.txt
%% and version 1.3 or later is part of all distributions of LaTeX
%% version 2005/12/01 or later.
%%
%% This work has the LPPL maintenance status `author-maintained'.
%%
%% This work consists of all files listed in README
%%
% Preamble
% We split the package loading and package option setting, because it's not a good idea to mix them, apparently: https://tex.stackexchange.com/a/745011/383565

% Packages
  % Encodings
  \usepackage[utf8]{inputenc}
  \usepackage[T1]{fontenc}
  % Language
  \usepackage[UKenglish]{babel} 
  % % Indentation
  %   % See posts: https://tex.stackexchange.com/a/39232/383565 and https://tex.stackexchange.com/a/39228/383565
  %   \usepackage{indentfirst}

  % Named colors
  \PassOptionsToPackage{dvipsnames}{xcolor}

  % Tables
    % % Set-up 1: LaTeX2e tabular-style tables
    % \usepackage{array, booktabs, multirow, cellspace}
    % \PassOptionsToPackage{table}{xcolor} % loads the colortbl package

    % Set-up 2 (recommended): LaTeX3 tabularray tables
    \usepackage{tabularray}
    \UseTblrLibrary{booktabs, zref}
      % Other useful tabularray libraries: 
        % amsmath: matrix, cases, and array environments with tabularray key=value options
        % diagbox: diagonal cells
        % siunitx: siunitx package loading + compatibility with tabularray

  % Figures
  \usepackage{graphicx}
  \ExplSyntaxOn
    \graphicspath{{\g_Prassble_path_to_Prassble_folder_tl/figures/}} 
  \ExplSyntaxOff
  \usepackage{caption, subcaption, float}

  % Math stuff
  \usepackage{amssymb, mathrsfs}
  \usepackage{mathtools, mleftright, nicematrix}
  \usepackage{intexgral}
    % prerequisites for \hatt command
    \usepackage{stackengine}
    \usepackage{verbatimbox} % For \addvbuffer

  % % Sciences
  % Add siunitx to \UseTblrLibrary{ <clist of libraries> } in the "Tables/Set-up 2" section of this file (preamble.tex)
  % For Chemical structure diagrams, check out the METAPOST package mcf2graph https://ctan.org/pkg/mcf2graph or the LaTeX package chemfig https://ctan.org/pkg/chemfig?lang=en

  % Diagrams
  \usepackage{tikz, pgfplots}
  \usetikzlibrary{cd}
  \pgfplotsset{compat=1.18}
  % \usepackage{asymptote}

  % enumeration
  \usepackage[shortlabels,inline]{enumitem}
  \usepackage{eqparbox}

  % Margins and line spacing
  \usepackage[margin = 1.25in]{geometry}
  \geometry{
    bottom=15mm}
  \usepackage{setspaceenhanced, parskip}
  \onehalfspacing

  % Page formatting
  \usepackage[pagestyles]{titlesec}
  \usepackage{lmodern, microtype}
    % default color theme
    \ifundef{\ColorTheme}{
      \gdef\ColorTheme{colourful}
    }{}
    % 'documentation' page style -> use 'main' page style with some extra options (detected by defining \PrassbleDocumentationStyle)
    \ifdefstring{\Pagestyle}{documentation}{
      \gdef\Pagestyle{main}
      \gdef\PrassbleDocumentationStyle{}
    }{}

  % Colored boxes
  \usepackage[skins, breakable, minted]{tcolorbox}
  \ifdef{\PrassbleDocumentationStyle}{
    \tcbuselibrary{documentation}
    \tcbset{listing engine=minted}
  }{}
    % Required for autowidth = lower 
    \usepackage{varwidth}
  \setminted{%
    ,escapeinside = ⵌⵌ
    % The characters afterwhich a linebreak can occur:
      % The backslashes are there only to escape the square brackets; the symbols "[" and "]" are the ones afterwhich a linebreak can occur. The \[\] is not indicating display mathmode.
      ,breakafter = \[\]()
  }
  \usepackage{keytheorems}

  % Externalisation 
  % \usepackage{robust-externalize}

  % Refer
  \PassOptionsToPackage{hyphens}{url}
  \usepackage{hyperref}
  \usepackage{zref-titleref}
  \usepackage{zref-clever}
  \usepackage{zref-xr}

% Custom setup

  % Easy input of files
  %% The LaTeX project Prassble
%% Input_macro.tex: an original command \Input for easy input of files.
%%
%% -------------------------------------------------------------------------------------------
%% Copyright (c) 2025 by Grass (GrassGlass) <shaohong00002 at gmail dot com>
%% -------------------------------------------------------------------------------------------
%%
%% This work may be distributed and/or modified under the
%% conditions of the LaTeX Project Public License, either version 1.3
%% of this license or (at your option) any later version.
%% The latest version of this license is in
%%   http://www.latex-project.org/lppl.txt
%% and version 1.3 or later is part of all distributions of LaTeX
%% version 2005/12/01 or later.
%%
%% This work has the LPPL maintenance status `author-maintained'.
%%
%% This work consists of all files listed in README
%%
\ExplSyntaxOn
% Easy input of files; \Input{#2} is \input{#2} but with the file path taken relative to
  % 1. the directory preamble, if \Input is used before \begin{document}
  % 2. the parent directory of the folder preamble (which in this case is the directory Prassble), if \Input is used after \begin{document}
  % 3. \CurrentFilePath, if (at anywhere) the starred variant \Input* is used
% \Input[#1]{#2} has the optional argument #1, which can prepend a file path #1 (folder_1/folder_2/.../folder_n) to the file path #2. That is, \Input[#1]{#2} = \input{#1#2}. But, #1 can also be a 'keyword' that causes the corresponding file path (defined with \KeywordforInput) to be prepended (instead of #1 itself), as well as cause some code (stored in \g_Prassble_Input_prepend_#1_tl and \g_Prassble_Input_append_#1_tl) to be possibly executed.
%* TLDR: \Input[ <keyword / file path to prepend> ]{ <file path relative to directory 'preamble' or '../preamble'> }
\clist_new:N \l__Prassble_files_to_be_input_clist
\seq_new:N \g__Prassble_Input_keywords_seq
\tl_new:N \g_Prassble_Input_prepend_tl
\tl_new:N \g_Prassble_Input_append_tl
\tl_new:N \g_file_directory_to_be_used_tl
\tl_new:N \g_Prassble_path_to_preamble_folder_tl
\tl_new:N \g_Prassble_path_to_Prassble_folder_tl
\tl_new:N \l__Prassble_relative_path_to_prepend_tl
% Set the directory to be the folder 'preamble' in the preamble.
  \tl_gset:Ne \g_Prassble_path_to_preamble_folder_tl {\CurrentFilePath/../../} %* The number of /../ here depends on where Input_macro.tex is placed inside the folder preamble
  \tl_gset:Ne \g_Prassble_path_to_Prassble_folder_tl {\g_Prassble_path_to_preamble_folder_tl../}
  \tl_set_eq:NN \g_file_directory_to_be_used_tl \g_Prassble_path_to_preamble_folder_tl
% Set the directory to be '../preamble' in the document body.
  \AtEndPreamble{\tl_gset_eq:NN \g_file_directory_to_be_used_tl \g_Prassble_path_to_Prassble_folder_tl}

\makeatletter
% Putting the core \Input code in a general function \__Prassble_Input@core:nnnn so that other commands similar to \input (e.g. \inputcode) can also be transformed accordingly (e.g. to \InputCode).
\cs_new:Npn \__Prassble_Input@core:nnnn #1#2#3#4 {
  % If the starred variant \Input* is used, go to the directory \CurrentFilePath.
    \bool_if:nT {#1} 
      {
        \tl_set:Nn \g_file_directory_to_be_used_tl {\CurrentFilePath}
      }
  % Test whether #2 is a keyword
    \seq_if_in:NeTF \g__Prassble_Input_keywords_seq {#2} 
      {
        % If so, set the corresponding 
          % path to prepend
          % code to prepend
          % code to append
        \tl_if_empty:cF {g__Prassble_Input_relative_path_to_prepend_#2_tl} {
          \tl_put_right:cn {g__Prassble_Input_relative_path_to_prepend_#2_tl} {/}
        }
        \tl_set_eq:Nc \l__Prassble_relative_path_to_prepend_tl {g__Prassble_Input_relative_path_to_prepend_#2_tl}
        \tl_set_eq:Nc \g_Prassble_Input_prepend_tl {g_Prassble_Input_prepend_#2_tl}
        \tl_set_eq:Nc \g_Prassble_Input_append_tl {g_Prassble_Input_append_#2_tl}
      } 
      {
        % Otherwise, #2 is a file path that is to prepended as-is
        \tl_if_empty:nF {#2} {
          \tl_set:Nn \l__Prassble_relative_path_to_prepend_tl {#2/}
        }
      }
    % \input the files approperiately
    \clist_set:Nn \l__Prassble_files_to_be_input_clist {#3} 
    \clist_map_inline:Nn \l__Prassble_files_to_be_input_clist 
        {
            \tl_use:N \g_Prassble_Input_prepend_tl
            #4{\g_file_directory_to_be_used_tl\l__Prassble_relative_path_to_prepend_tl##1}
            \tl_use:N \g_Prassble_Input_append_tl
        }
  % Reset the path to prepend
    \tl_clear:N \l__Prassble_relative_path_to_prepend_tl
  % Reset the keyword
  \tl_clear:N \l__Prassble_relative_path_to_prepend_tl
  \tl_clear:N \g_Prassble_Input_prepend_tl
  \tl_clear:N \g_Prassble_Input_append_tl
}
\NewDocumentCommand{\Input}{ s O{} m }{
  \__Prassble_Input@core:nnnn { #1 } { #2 } { #3 } { \input }
}
\makeatother
\ExplSyntaxOff
  \Input{commands/modified_commands/Chapter_and_Section.tex}
  \Input{commands/newcommands/KeywordForInput.tex}
  \KeywordForInput{HW}{files/homework}                %*
  \KeywordForInput{Ch}{files/chapters}                %*
  
  % Math stuff
  \Input[math]{%
    ,general_math.tex%
    ,specific_math.tex%
    }
  
  % enumeration
  \Input{enumeration.tex}

  % Custom symbols (\eeveeKawaii, \TikZ, etc)
  \Input{custom-symbols.tex}
  
  % page formatting
  \definecolor{HeaderColor}{RGB}{0,82,155}            %*
  \author{Author}                                     %*
  \def\CourseName{\textlangle Course Name\textrangle} %*
  \newcounter{HWNumber}                               %*
  \Input{page-formatting.tex}
  \Input{hw_zexternaldocument_main.tex}
  \AddToHWEnvList{table,figure}                       %*

  % Colored boxes
    % Nonlistings
    \Input[environments/box_styles/nonlistings]{%
      ,theoremstyles.tex%
      ,commands/modified_commands/newkeytheorem.tex
      ,../lengths_and_counters.tex%
    }
    \Input[environments/derived_environments]{%
      ,lengths.tex%
      ,nonlistings.tex%
    }
    % Listings
    \Input[environments/box_styles/listings/commands/newcommands]{%
      ,all_short_names.tex%
      ,new_variables.tex%
      ,name_converters.tex%
    }
    \Input[environments/box_styles/listings/commands/modified_commands]{%
      ,../../codestyles.tex%
      ,NewInputListing.tex%
      ,NewListing.tex%
    }
    \Input{environments/derived_environments/listings.tex}
    \Input{environments/box_styles/nonlistings/commands/newcommands/InputKeyword_HW-in-Main.tex}

  % Refer
  \Input{external_hyperlink_format.tex}
  \hypersetup{
    ,colorlinks=true
    ,linkcolor=blue
    ,filecolor=magenta
    ,urlcolor=blue
    }

  % Other
  \Input{other.tex}
  \newcommand{\HWInMainHeader}{\subsection{Homework \#\arabic{HWNumber}}}                                        %*
\usepackage[edges]{forest}
\begin{document}
% Set the current file directory to 'Documentation' --- we basically execute, 'cd "...Prassble/Documentation"' 
\ExplSyntaxOn
    \tl_gset:Ne \g_file_directory_to_be_used_tl {}
\ExplSyntaxOff
\frontmatter
\Input{Doc_Titlepage.tex}
\thispagestyle{empty} 
%
\chapter{Preface: What Is This Project?}
This aims to be a beautiful coloured {\LaTeX} template for everyone to enjoy, for taking notes, writing a book, and more. 

But there are so many {\LaTeX} templates available online already, why another? This started as a personal project to revise my {\LaTeX} setup, to make it more convenient and hassle free, as well as to improve the aesthetics of my documents. Over time, I refined this template to what it is today and wanted everyone to be able to enjoy its use; naturally, I'm releasing it under the \extref{https://www.latex-project.org/lppl/}{LPPL 1.3}. (Moreover, coloured templates seem unfortunately rare?)

I have learnt a bit more about {\LaTeX} programming over the course of this project, but I'm still a mere amateur. So, while feedback is greatly appreciated, please don't be too harsh on me when you spot some abominable code --- I try to write decent {\TeX} and update it when I find out bad practices I engage in.
%
\addtocontents{toc}{\protect\thispagestyle{empty}}
\tableofcontents
\thispagestyle{empty}
\newpage
\setcounter{page}{1}
%
\mainmatter
\Chapter{The Contents of This Template}
While the \extref{https://github.com/GrassGlass/Prassble}{GitHub repository} of Prassble contains other files/folders for development purposes, the core template consists of only these:
\definecolor{folderbg}{RGB}{124,166,198}
\definecolor{folderborder}{RGB}{110,144,169}
\newlength\Size
\setlength\Size{4pt}
\tikzset{%
  folder/.pic={%
    \filldraw [draw=folderborder, top color=folderbg!50, bottom color=folderbg] (-1.05*\Size,0.2\Size+5pt) rectangle ++(.75*\Size,-0.2\Size-5pt);
    \filldraw [draw=folderborder, top color=folderbg!50, bottom color=folderbg] (-1.15*\Size,-\Size) rectangle (1.15*\Size,\Size);
  },
  file/.pic={%
    \filldraw [draw=folderborder, top color=folderbg!5, bottom color=folderbg!10] (-\Size,.4*\Size+5pt) coordinate (a) |- (\Size,-1.2*\Size) coordinate (b) -- ++(0,1.6*\Size) coordinate (c) -- ++(-4pt,5pt) coordinate (d) -- cycle (d) |- (c) ;
  },
}
\forestset{%
  declare autowrapped toks={pic me}{},
  declare boolean register={pic root},
  pic root=0,
  pic dir tree/.style={%
    for tree={%
      folder,
      font=\sffamily,
      grow'=0,
    },
    before typesetting nodes={%
      for tree={%
        edge label+/.option={pic me},
      },
      if pic root={
        tikz+={
          \pic at ([xshift=\Size].west) {folder};
        },
        align={l}
      }{},
    },
  },
  pic me set/.code n args=2{%
    \forestset{%
      #1/.style={%
        inner xsep=2\Size,
        pic me={pic {#2}},
      }
    }
  },
  pic me set={directory}{folder},
  pic me set={file}{file},
}
\begin{center}
\begin{forest}
  pic dir tree,
  pic root,
  for tree={% folder icons by default; override using file for file icons
    ,directory
    ,fit = band
    ,l = 0pt
    ,s sep = 0pt
  },
  [Prassble
    [.devcontainer
      [devcontainer.json, file]
      [Dockerfile       , file]
      [LICENSE          , file]
    ]
    [.vscode
      [settings.json    , file]    
    ]
    [preamble,
      []
    ]
  ]
\end{forest}
\end{center}
%
\backmatter
\end{document}