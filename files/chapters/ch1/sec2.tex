\subsection{Theorems}
\begin{theorem}
    This is the second theorem.
\end{theorem}
\begin{proof}
    This is the proof to the second theorem.
\end{proof}
\begin{corollary}(standalone)   
    This is a corollary that is trivial enough for the proof to be left unstated. As such, we use the \TeXinline{type in thmgroup = standalone} option.
\end{corollary}
\subsection{Exercises}
\begin{exercise}[label=ex:1.2]
    Another exercise.
\end{exercise}
\begin{code}{Text}
    Some code as part of the answer to ⵌ\zcref{ex:1.2}ⵌ.
\end{code}
    % Does tcolorbox's transparency option work in breakable tcolorboxes? Yes.ⵌ
    % Does tcolorbox's transparency option work in broken tcolorboxes? Yes.
    % 
    % 
    % 
    % 
    % 
    % 
    % 
    % 
    % 
    % 
    % 
    % 
    % 
    % 
    % 
\begin{answer}
    The answer to \zcref{ex:1.2}.
\end{answer}
These are some hyperlinks with \TeXinline{zref-clever}: \zcref{ex:1.1}, \zcref{ex:1.2}, \zcref{ex:H.1.1}, \zcref{ex:H.1.2}, \zcref{ex:H.1.3}. We can also reference theorems: \zcref{thm:1.1}. Furthermore, \extref{https://www.google.com/search?q=external+link}{external links} are nicely formatted with the \TeXinline{\extref{ⵌ\textlangleⵌURLⵌ\textrangleⵌ}{ⵌ\textlangleⵌlink nameⵌ\textrangleⵌ}} command.
\newpage
This is the second page.

\newpage
This is the third page.