% Source - https://tex.stackexchange.com/a/521559
% Posted by user194703, modified by community. See post 'Timeline' for change history
% Retrieved 2026-02-14, License - CC BY-SA 4.0

\documentclass[tikz,margin=3]{standalone}
\usetikzlibrary{fpu}
\newcommand{\pgfmathparseFPU}[1]{\begingroup%
\pgfkeys{/pgf/fpu,/pgf/fpu/output format=fixed}%
\pgfmathparse{#1}%
\pgfmathsmuggle\pgfmathresult\endgroup}
\makeatletter
\pgfmathdeclarefunction{distance}{2}{%
\begingroup%
\pgfextractx{\pgf@xa}{\pgfpointanchor{#1}{center}}%
\pgfextracty{\pgf@ya}{\pgfpointanchor{#1}{center}}%
\pgfextractx{\pgf@xb}{\pgfpointanchor{#2}{center}}%
\pgfextracty{\pgf@yb}{\pgfpointanchor{#2}{center}}%
\pgfmathparseFPU{sqrt((\pgf@xa-\pgf@xb)*(\pgf@xa-\pgf@xb)+(\pgf@ya-\pgf@yb)*(\pgf@ya-\pgf@yb))}%
\pgfmathsmuggle\pgfmathresult\endgroup%
}%
\makeatother
\begin{document}
\begin{tikzpicture}
  \draw (0,0) coordinate (b) -- (40pt,0)
              coordinate (c) -- (40pt,30pt)
              coordinate (a) -- cycle;
  \pgfmathsetmacro{\mydistance}{distance("a","b")}            
  \path node 
  {$\pgfmathprintnumber{\mydistance}\,\mathrm{pt}=
  \pgfmathparse{\mydistance/1cm}\pgfmathprintnumber{\pgfmathresult}\,\mathrm{cm}$};
  \draw (0,0) circle [radius={distance("a","b")}];
\end{tikzpicture}
\end{document}
