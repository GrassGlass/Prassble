\documentclass{book}
% %% The LaTeX project Prassble
%% preamble.tex: the preamble of Prassble
%%
%% -------------------------------------------------------------------------------------------
%% Copyright (c) 2025 by Grass (GrassGlass) <shaohong00002 at gmail dot com>
%% -------------------------------------------------------------------------------------------
%%
%% This work may be distributed and/or modified under the
%% conditions of the LaTeX Project Public License, either version 1.3
%% of this license or (at your option) any later version.
%% The latest version of this license is in
%%   http://www.latex-project.org/lppl.txt
%% and version 1.3 or later is part of all distributions of LaTeX
%% version 2005/12/01 or later.
%%
%% This work has the LPPL maintenance status `author-maintained'.
%%
%% This work consists of all files listed in README
%%
% Preamble
% We split the package loading and package option setting, because it's not a good idea to mix them, apparently: https://tex.stackexchange.com/a/745011/383565

% Packages
  % Encodings
  \usepackage[utf8]{inputenc}
  \usepackage[T1]{fontenc}
  % Language
  \usepackage[UKenglish]{babel} 
  % % Indentation
  %   % See posts: https://tex.stackexchange.com/a/39232/383565 and https://tex.stackexchange.com/a/39228/383565
  %   \usepackage{indentfirst}

  % Named colors
  \PassOptionsToPackage{dvipsnames, table}{xcolor} 
    % The table option of xcolor loads the colortbl package, in order to use the tools for coloring rows, columns, and cells within tables.

  % Tables
  \usepackage{array, booktabs, multirow, cellspace}

  % Figures
  \usepackage{graphicx}
  \graphicspath{{\CurrentFilePath/../figures/}} 
  \usepackage{caption, subcaption, float}

  % Math stuff
  \usepackage{amssymb, mathrsfs}
  \usepackage{mathtools, mleftright, nicematrix}
  \usepackage{intexgral}
    % prerequisites for \hatt command and \reallywidehat
    \usepackage{scalerel, stackengine}
    \usepackage{verbatimbox} % For \addvbuffer

  % % Sciences
  % \usepackage{siunitx}

  % Diagrams
  \usepackage{tikz, pgfplots}
  \usetikzlibrary{cd}
  \pgfplotsset{compat=1.18}

  % enumeration
  \usepackage[shortlabels,inline]{enumitem}
  \usepackage{eqparbox}

  % Margins and line spacing
  \usepackage[margin = 1.25in]{geometry}
  \geometry{
    bottom=15mm}
  \usepackage{setspaceenhanced, parskip}
  \onehalfspacing

  % Page formatting
  \usepackage[pagestyles]{titlesec}
  \usepackage{lmodern, microtype}
    % default colour theme
    \ifundef{\ColourTheme}{\gdef\ColourTheme{colourful}}{}

  % Colored boxes
  \usepackage[skins, breakable, minted]{tcolorbox}
    % Required for autowidth = lower 
    \usepackage{varwidth}
  \setminted{%
    ,escapeinside = ⵌⵌ
    % The characters afterwhich a linebreak can occur:
      % The backslashes are there only to escape the square brackets; the symbols "[" and "]" are the ones afterwhich a linebreak can occur. The \[\] is not indicating display mathmode.
      ,breakafter = \[\]()
  }
  \usepackage{keytheorems}

  % Externalisation 
  % \usepackage{robust-externalize}

  % Refer
  \usepackage{soul} % for special underlining of hyperlinks --- descenders are omitted
  \PassOptionsToPackage{hyphens}{url}
  \usepackage{hyperref}
  \usepackage{zref-titleref}
  \usepackage{zref-clever}
  \usepackage{zref-xr}

% Custom setup

  % Easy input of files
  %% The LaTeX project Prassble
%% Input_macro.tex: an original command \Input for easy input of files.
%%
%% -------------------------------------------------------------------------------------------
%% Copyright (c) 2025 by Grass (GrassGlass) <shaohong00002 at gmail dot com>
%% -------------------------------------------------------------------------------------------
%%
%% This work may be distributed and/or modified under the
%% conditions of the LaTeX Project Public License, either version 1.3
%% of this license or (at your option) any later version.
%% The latest version of this license is in
%%   http://www.latex-project.org/lppl.txt
%% and version 1.3 or later is part of all distributions of LaTeX
%% version 2005/12/01 or later.
%%
%% This work has the LPPL maintenance status `author-maintained'.
%%
%% This work consists of all files listed in README
%%
\ExplSyntaxOn
% Easy input of files; \Input{#2} is \input{#2} but with the file path taken relative to
  % 1. the directory preamble, if \Input is used before \begin{document}
  % 2. the parent directory of the folder preamble (which in this case is the directory Prassble), if \Input is used after \begin{document}
  % 3. \CurrentFilePath, if (at anywhere) the starred variant \Input* is used
% \Input[#1]{#2} has the optional argument #1, which can prepend a file path #1 (folder_1/folder_2/.../folder_n) to the file path #2. That is, \Input[#1]{#2} = \input{#1#2}. But, #1 can also be a 'keyword' that causes the corresponding file path (defined with \KeywordforInput) to be prepended (instead of #1 itself), as well as cause some code (stored in \g_Prassble_Input_prepend_#1_tl and \g_Prassble_Input_append_#1_tl) to be possibly executed.
%* TLDR: \Input[ <keyword / file path to prepend> ]{ <file path relative to directory 'preamble' or '../preamble'> }
\clist_new:N \l__Prassble_files_to_be_input_clist
\seq_new:N \g__Prassble_Input_keywords_seq
\tl_new:N \g_Prassble_Input_prepend_tl
\tl_new:N \g_Prassble_Input_append_tl
\tl_new:N \g_file_directory_to_be_used_tl
\tl_new:N \g_Prassble_path_to_preamble_folder_tl
\tl_new:N \g_Prassble_path_to_Prassble_folder_tl
\tl_new:N \l__Prassble_relative_path_to_prepend_tl
% Set the directory to be the folder 'preamble' in the preamble.
  \tl_gset:Ne \g_Prassble_path_to_preamble_folder_tl {\CurrentFilePath/../../} %* The number of /../ here depends on where Input_macro.tex is placed inside the folder preamble
  \tl_gset:Ne \g_Prassble_path_to_Prassble_folder_tl {\g_Prassble_path_to_preamble_folder_tl../}
  \tl_set_eq:NN \g_file_directory_to_be_used_tl \g_Prassble_path_to_preamble_folder_tl
% Set the directory to be '../preamble' in the document body.
  \AtEndPreamble{\tl_gset_eq:NN \g_file_directory_to_be_used_tl \g_Prassble_path_to_Prassble_folder_tl}

\makeatletter
% Putting the core \Input code in a general function \__Prassble_Input@core:nnnn so that other commands similar to \input (e.g. \inputcode) can also be transformed accordingly (e.g. to \InputCode).
\cs_new:Npn \__Prassble_Input@core:nnnn #1#2#3#4 {
  % If the starred variant \Input* is used, go to the directory \CurrentFilePath.
    \bool_if:nT {#1} 
      {
        \tl_set:Nn \g_file_directory_to_be_used_tl {\CurrentFilePath}
      }
  % Test whether #2 is a keyword
    \seq_if_in:NeTF \g__Prassble_Input_keywords_seq {#2} 
      {
        % If so, set the corresponding 
          % path to prepend
          % code to prepend
          % code to append
        \tl_if_empty:cF {g__Prassble_Input_relative_path_to_prepend_#2_tl} {
          \tl_put_right:cn {g__Prassble_Input_relative_path_to_prepend_#2_tl} {/}
        }
        \tl_set_eq:Nc \l__Prassble_relative_path_to_prepend_tl {g__Prassble_Input_relative_path_to_prepend_#2_tl}
        \tl_set_eq:Nc \g_Prassble_Input_prepend_tl {g_Prassble_Input_prepend_#2_tl}
        \tl_set_eq:Nc \g_Prassble_Input_append_tl {g_Prassble_Input_append_#2_tl}
      } 
      {
        % Otherwise, #2 is a file path that is to prepended as-is
        \tl_if_empty:nF {#2} {
          \tl_set:Nn \l__Prassble_relative_path_to_prepend_tl {#2/}
        }
      }
    % \input the files approperiately
    \clist_set:Nn \l__Prassble_files_to_be_input_clist {#3} 
    \clist_map_inline:Nn \l__Prassble_files_to_be_input_clist 
        {
            \tl_use:N \g_Prassble_Input_prepend_tl
            #4{\g_file_directory_to_be_used_tl\l__Prassble_relative_path_to_prepend_tl##1}
            \tl_use:N \g_Prassble_Input_append_tl
        }
  % Reset the path to prepend
    \tl_clear:N \l__Prassble_relative_path_to_prepend_tl
  % Reset the keyword
  \tl_clear:N \l__Prassble_relative_path_to_prepend_tl
  \tl_clear:N \g_Prassble_Input_prepend_tl
  \tl_clear:N \g_Prassble_Input_append_tl
}
\NewDocumentCommand{\Input}{ s O{} m }{
  \__Prassble_Input@core:nnnn { #1 } { #2 } { #3 } { \input }
}
\makeatother
\ExplSyntaxOff
  \KeywordForInput{HW}{files/homework}                %*
  
  % Math stuff
  \Input[math]{%
    ,general_math.tex%
    ,specific_math.tex%
    }
  
  % enumeration
  \Input{enumeration.tex}

  % Custom symbols (\eeveeKawaii, \TikZ, etc)
  \Input{custom-symbols.tex}
  
  % page formatting
  \definecolor{HeaderColour}{RGB}{0,82,155}            %*
  \author{Author}                                     %*
  \def\CourseName{\textlangle Course Name\textrangle} %*
  \newcounter{HWNumber}                               %*
  \Input{page-formatting.tex}

  % Colored boxes
    % Nonlistings
    \Input[environments/box_styles/nonlistings]{%
      ,theoremstyles.tex%
      ,commands/modified_commands/newkeytheorem.tex
      ,commands/newcommands/InputKeyword_HW-in-Main.tex
      ,../lengths_and_counters.tex%
    }
    \Input[environments/derived_environments]{%
      ,lengths.tex%
      ,nonlistings.tex%
    }
    % Listings
    \Input[environments/box_styles/listings/commands/newcommands]{%
      ,all_short_names.tex%
      ,new_variables.tex%
      ,name_converters.tex%
    }
    \Input[environments/box_styles/listings/commands/modified_commands]{%
      ,../../codestyles.tex%
      ,NewInputListing.tex%
      ,NewListing.tex%
    }
    \Input{environments/derived_environments/listings.tex}

  % Refer
  \Input{external_hyperlink_format.tex}
  \hypersetup{
    ,colorlinks=true
    ,linkcolor=blue
    ,filecolor=magenta
    ,urlcolor=cyan
    }

  % Other
  \Input{other.tex}
  \AddToHWEnvList{table,figure}                       %*
  \newcommand{\HWInMainHeader}{\subsection{Homework \#\arabic{HWNumber}}}                                        %*
\usepackage[edges]{forest}
\definecolor{folderbg}{RGB}{124,166,198}
\definecolor{folderborder}{RGB}{110,144,169}
\newlength\Size
\setlength\Size{4pt}
\tikzset{%
  folder/.pic={%
    \filldraw [draw=folderborder, top color=folderbg!50, bottom color=folderbg] (-1.05*\Size,0.2\Size+5pt) rectangle ++(.75*\Size,-0.2\Size-5pt);
    \filldraw [draw=folderborder, top color=folderbg!50, bottom color=folderbg] (-1.15*\Size,-\Size) rectangle (1.15*\Size,\Size);
  },
  file/.pic={%
    \filldraw [draw=folderborder, top color=folderbg!5, bottom color=folderbg!10] (-\Size,.4*\Size+5pt) coordinate (a) |- (\Size,-1.2*\Size) coordinate (b) -- ++(0,1.6*\Size) coordinate (c) -- ++(-4pt,5pt) coordinate (d) -- cycle (d) |- (c) ;
  },
}
\forestset{%
  declare autowrapped toks={pic me}{},
  declare boolean register={pic root},
  pic root=0,
  pic dir tree/.style={%
    for tree={%
      folder,
      font=\ttfamily,
      grow'=0,
    },
    before typesetting nodes={%
      for tree={%
        edge label+/.option={pic me},
      },
      if pic root={
        tikz+={
          \pic at ([xshift=\Size].west) {folder};
        },
        align={l}
      }{},
    },
  },
  pic me set/.code n args=2{%
    \forestset{%
      #1/.style={%
        inner xsep=2\Size,
        pic me={pic {#2}},
      }
    }
  },
  ,file tree/.style = {
    ,pic me set={directory}{folder}
    ,pic me set={file}{file}
    ,pic dir tree
    ,pic root
    ,for tree={% folder icons by default; override using file for file icons
      ,directory
      ,fit = band
      ,l sep = 2\Size+2mm
    }
  }
}
\begin{document}
Hi
\ExplSyntaxOn
  \str_new:N \g__Prassble_filetree_i_str
  \str_new:N \g__Prassble_filetree_I_only_str
  \str_new:N \g__Prassble_filetree_file_path_str
  \str_new:N \g__Prassble_filetree_final_i_str
  \str_new:N \g__Prassble_filetree_final_I_str
  \str_new:N \g__Prassble_filetree_Output_To_str
  \str_new:N \g__Prassble_filetree_final_Output_To_str
  \int_new:N \g__Prassble_filetree_current_tree_number_int
  \bool_new:N \g__Prassble_filetree_I_bool
    \msg_new:nnn {Prassble_filetree} {shell_escape_is_disabled} {(\msg_line_context:)~Unrestricted~shell~escape~is~disabled.~Using~cached~files,~if~they~exist.}
  \msg_new:nnn {Prassble_filetree} {missing_file_tree_.tex} {The~file~#1~is~not~found~\msg_line_context:;~compile~with~unrestricted~shell~escape~to~generate~it.}
  \keys_define:nn {Prassble_filetree} {
    ,i              .str_set:N = \g__Prassble_filetree_i_str
      ,ignore         .meta:n   = { i = #1 }


    ,i+             .code:n   = { \str_put_right:Nn { \g__Prassble_filetree_i_str } }
      ,ignore-add     .meta:n   = { i+ = #1 }
      ,ignore-also    .meta:n   = { i+ = #1 }


    ,i-extensions   .code:n   = { 
      \clist_map_inline:nn { #1 } {
        % Make sure each item (file extension) of the user's clist is prepended with a dot (if it isn't already there). Save this new clist under \l_tmpa_clist. 
        \str_if_eq:nnTF 
          { \str_item:nn {##1} {1} } 
          { . }
          { \clist_put_right:Nn \l_tmpa_clist { ##1 } }
          { \clist_put_right:Nn \l_tmpa_clist { .##1 } }
      }
      \str_put_right:Ne \g__Prassble_filetree_i_str { \clist_use:Nn \l_tmpa_clist { | } }
    }


    ,i-match-name   .code:n   = {
      \str_put_right:Ne \g__Prassble_filetree_i_str { \clist_use:nn { #1 } { | } }
    }
      ,i-folders      .meta:n   = { i-match-name = #1 }
      ,i-files        .meta:n   = { i-match-name = #1 }


    ,i-clear        .code:n   = { \str_clear:N \g__Prassble_filetree_i_str }

    ,I              .code:n   = {
      \bool_set_true:N \g__Prassble_filetree_I_bool
      \str_set:Nn \g__Prassble_filetree_I_only_str {#1}
    }
      ,Ignore         .meta:n   = { I = #1 }
      ,Ignore-only    .meta:n   = { I = #1 }
      ,I-only         .meta:n   = { I = #1 }
    ,I-clear        .code:n   = { \str_clear:N \g__Prassble_filetree_I_only_str }

    
    ,O              .str_set:N   = \g__Prassble_filetree_Output_To_str
      ,output-to      .meta:n   = { O = #1 }
      ,output-To      .meta:n   = { O = #1 }
      ,Output-to      .meta:n   = { O = #1 }
      ,Output-To      .meta:n   = { O = #1 }
  }
  
  \NewDocumentCommand{\filetree}{ O{} m }{
  % Step the counter for the file tree number
    \int_incr:N \g__Prassble_filetree_current_tree_number_int
  % Default output location
    \str_set:Ne \g__Prassble_filetree_Output_To_str {FileTrees/tree_\int_use:N \g__Prassble_filetree_current_tree_number_int.tex}
    \group_begin:
      \keys_set:nn { Prassble_filetree } { #1 }
      % Final flag processing before parsing to file_tree.sh
        % -i
        \str_if_empty:NF \g__Prassble_filetree_i_str {
          \str_set:Nn \g__Prassble_filetree_final_i_str { -i~"\g__Prassble_filetree_i_str" }
        }
      % -I
        \bool_if:NT \g__Prassble_filetree_I_bool {
          \str_set:Nn \g__Prassble_filetree_final_I_str { -I~"\g__Prassble_filetree_I_only_str"}
        }
      % -O
        \str_if_empty:NF \g__Prassble_filetree_Output_To_str {
          \str_set:Ne \g__Prassble_filetree_final_Output_To_str { -O~"\g__Prassble_filetree_Output_To_str" }
        }
        
      \str_set:Ne \g__Prassble_filetree_file_path_str {
        \str_if_empty:nTF { #2 }
          { ./ }
          { #2 }
      }
      \sys_shell_now:e {bash~file_tree.sh\space\str_use:N \g__Prassble_filetree_final_i_str\space\str_use:N \g__Prassble_filetree_final_I_str\space\str_use:N \g__Prassble_filetree_final_Output_To_str\space\str_use:N \g__Prassble_filetree_file_path_str}
    \group_end:
    \input{FileTrees/tree_\int_use:N \g__Prassble_filetree_current_tree_number_int.tex}
  }

% \makeastretter
  % \str_new:N \l__Prassble_FileTree_full_path_str
  % \NewDocumentCommand{\FileTree}{ s O{} m }{
  %   \__Prassble_Input@core:nnnn { #1 } { #2 } { #3 } { \str_set:Ne \l__Prassble_FileTree_full_path_str }
  %   \filetree{\l__Prassble_FileTree_full_path_str}
  % }
% \makeatother
\ExplSyntaxOff
\begin{center}
  \filetree{some_folder}
  \filetree{../tree}
  \filetree[i=folder]{}
  \filetree{../../preamble/commands/modified_commands/../../../}
\end{center}
\newpage
\begin{center}
  \filetree[I=]{./}
  \filetree{}
\end{center}
\newpage
% Here: \CurrentFilePath\\
% \begin{center}
%   \FileTree{testing/tree}
% \end{center}
\newpage

\end{document}
