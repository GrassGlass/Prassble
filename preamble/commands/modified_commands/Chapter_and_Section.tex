\ExplSyntaxOn
    % Store the chapter/section numbers as text integers.
        \tl_new:N \g__prassble_chapter_text_number_tl
    % Command \Chapter that inserts a \chapter header and automatically creates the corresponding keyword for input Ch\arabic{chapter}, that changes the directory to that of the current chapter --- the subdirectory chapters/ch\arabic{chapter} of Prassble/files.
        \NewDocumentCommand{\Chapter}{ m }{
            % Usual chapter header
                \chapter{#1}
            % Change the directory to that of the current chapter (Prassble/files/chapters/ch\arabic{chapter})
                % \tl_set:Ne is used to expand \arabic{chapter} into text form.
                    \tl_set:Ne \g__prassble_chapter_text_number_tl {\arabic{chapter}}
                % \tl_map_inline is used to extract the value of \arabic{chapter} in text form so that the keyword is Ch1 (or Ch2, etc) instead of literally "Ch\arabic{chapter}"
                    \tl_map_inline:Nn \g__prassble_chapter_text_number_tl {\KeywordForInput{Ch##1}{files/chapters/ch##1}}
        }
    % Comamnd \Section that inserts a \section header, and then, automatically \Input's the corresponding sec\arabic{section}.tex file immediately after.
        \NewDocumentCommand{\Section}{ m }{
        % Usual section header
                \section{#1}
            % Change the directory to that of the current chapter (Prassble/files/chapters/ch\arabic{chapter})
                % \tl_set:Ne is used to expand \arabic{chapter} into text form.
                    \tl_set:Ne \g__prassble_chapter_text_number_tl {\arabic{chapter}}
                % \tl_map_inline is used to extract the value of \arabic{chapter} in text form so that the keyword is Ch1 (or Ch2, etc) instead of literally "Ch\arabic{chapter}"
                    \tl_map_inline:Nn \g__prassble_chapter_text_number_tl {\Input[Ch##1]{sec\arabic{section}.tex}}
        }
\ExplSyntaxOff