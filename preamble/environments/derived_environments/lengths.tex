% boxsep
\setlength{\boxsep}{1pt}

% boxrule
\setlength{\boxrule}{0pt}

% The approximate vertical space between two consequective (non-listing) colorboxes.
    % \vspace{-\vlengthBetweenColoredBoxes} is used to flush two consequctive (*non listing*) colorboxes together, such that there is no whitespace between them.
    \setlength{\vlengthBetweenColoredBoxes}{9.1pt}

% The thickness of the western borderline (i.e. on the left boundary) of the colorboxes
\setlength{\BorderlineWestThickness}{2pt}

% The approximate vertical space from the top of a listing box to the bottom of another (non-listing) tcolorbox --- placed directly above the listing box.
\setlength{\vlengthbeforelisting}{3.6pt}

% The approximate vertical space from the bottom of a listing box to the top of another (non-listing) tcolorbox --- placed directly below the listing box.
\setlength{\vlengthafterlisting}{3.4375pt}

% Corrects the vertical space before starred listing environments
    % The approximate vertical space from the top of a listing box to the bottom of another (non-listing) tcolorbox --- placed directly above the listing box
    % is made the same as
    % The approximate vertical space from the top of a (non-listing) tcolorbox to the bottom of another (non-listing) tcolorbox --- placed directly above the first.
    \setlength{\vlengthbetweencorrector}{7.12pt}

% The distance between each code line number and the verbatim code on the same line 
    % More accurately, in this case, rather than ``the number'', it is 
    % \eqmakebox[code\arabic{code}][c]{\textcolor{Maroon}{\scriptsize\arabic{FancyVerbLine}}})
    \setlength{\numbersep}{0.2474691189148802em}

% Controls the length of the code line number backgrounds
    % The code line number overlays have a background whose color is darker than the rest of the listing box, but lighter than the borderline to the left. This background has horizontal width \eqboxwidth{code\arabic{code}} + \linenumberpadding
    \setlength{\linenumberpadding}{0.2474691189148802em}

%* The maximum length that the listing boxes can otherwise overflow by 
    % Depends on the document's textwidth --- the length \edef\textwidthoutside{\the\textwidth} placed in the preamble
    %! Pick a listing environment (of maximal local width, i.e. with "breaklines" enabled) and let the local textwidth inside it be \localmaxtextwidth. It is *extremely* important that \localmaxtextwidth+\eqboxwidth{codewidth\arabic{code}}+\lefthspace+\linebreakcorrector => \textwidthoutside, for the autowidth to work correctly. 
        %* Otherwise, we will face a bug where deleting the .aux file is necessary when recompiling the listing environment, after changes have been made past the initial two compilations.
    % Ideally, just ever so slightly less than, so that breaklines isn't automatically triggered when the box --- for which the optional argument ' or " has been activated --- can still fit within the bounds of your page (i.e. box width => \textwidthoutside).
    \setlength{\linebreakcorrector}{13.38109pt}

% The length to move the code line numbers right by, such that they are horizontally centered on their background.
\setlength{\lefthspace}{8.5pt}
    
% How much to move the title right by, to make the title's positioning consistent with other colorboxes
\setlength{\titleleft}{1.1pt}