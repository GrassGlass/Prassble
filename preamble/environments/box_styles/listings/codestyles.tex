%% The LaTeX project Prassble
%% codestyles.tex: the tcbkey (pgfkey) that sets the style of the listing boxes (tcblisting/tcbinputlisting options)
%%
%% -------------------------------------------------------------------------------------------
%% Copyright (c) 2025 by Grass (GrassGlass) <shaohong00002 at gmail dot com>
%% -------------------------------------------------------------------------------------------
%%
%% This work may be distributed and/or modified under the
%% conditions of the LaTeX Project Public License, either version 1.3
%% of this license or (at your option) any later version.
%% The latest version of this license is in
%%   http://www.latex-project.org/lppl.txt
%% and version 1.3 or later is part of all distributions of LaTeX
%% version 2005/12/01 or later.
%%
%% This work has the LPPL maintenance status `author-maintained'.
%%
%% This work consists of all files listed in README
%%
% User level comparator
    % source: https://tex.stackexchange.com/a/748134/383565
\makeatletter
\ExplSyntaxOn
\cs_set_eq:NN \DimComparenNnT \dim_compare:nNnT
\cs_set_eq:NN \DimComparenNnTF \dim_compare:nNnTF
\cs_set_eq:NN \BoolLazyOrnnT \bool_lazy_or:nnT
\cs_set_eq:NN \BoolLazyOrnnTF \bool_lazy_or:nnTF
\cs_set_eq:NN \TLIfEqeeT \tl_if_eq:eeT

% Colors
\def\ColorCodeBase{Maroon}
\def\ColorCodeColback{\ColorCodeBase!5}
\def\ColorCodeColbacktitle{\ColorCodeBase!5}
\def\ColorCodeTitle{\ColorCodeBase!70!black}
\def\ColorCodeBorderline{\ColorCodeBase}
\def\ColorCodeLineNumber{\ColorCodeBase}
\def\ColorCodeLineNumberBackground{\ColorCodeBase!20}

% Variables
  % Definitions
  \tl_new:N \g__listings_linenos_tl
  \tl_new:N \g__listings_breaklines_tl
  \cs_new:Npn \__listings_title_contents:n #1 {#1 \g__listings_OTitle_tl.}
  \tl_new:N \g__listings_OTitle_tl
  \dim_new:N \l__listings_autowidth_title_dim
  \dim_new:N \l__listings_autowidth_upper_dim
  \dim_new:N \l__listings_autowidth_lower_dim
  \dim_new:N \g__listings_autowidth@core_dim
  \cs_new:Npn \__tcb_text_width_to_width:n #1
  {
    #1+\kvtcb@left@rule+\kvtcb@right@rule+\kvtcb@boxsep*2+\kvtcb@leftupper+\kvtcb@rightupper
  }
  \cs_new:Npn \__listings_autowidth_title:n #1
  {
    \bool_if:NT 
      {\g__listings_autowidth_title_bool} 
      {
        \eqsavebox{\codewidth}[codetitlewidth\arabic{code}]
        {
          \__listings_title_contents:n {#1}
        }
      }
  }
  \cs_new:Npn \__listings_autowidth_lower_eqsavebox:n #1
  {
    \bool_if:NT
      {\g__listings_autowidth_lower_bool}
      {
        \sbox0=\hbox{\eqparbox{codelowerwidth\arabic{code}}{#1}}
      }
  }
  \cs_new:Npn \__listings_debug_autowidth:
  {
    \bool_if:NT \g__listings_debug_autowidth_bool
        { 
          \renewcommand{\labelalignment}{c}
          \begin{enumerate}
            \item \emph{\textbackslash arabic\{code\}} \({}={}\) \arabic{code}
            \item \emph{\textbackslash eqboxwidth\{\__listings_debug_argument_formatter:n {tag}\}}
              \begin{enumerate}
                \item
                [
                    \underline
                    {
                      \__listings_debug_argument_formatter:n {tag}
                    }
                  ]  
                \item[codetitlewidth:] \eqboxwidth{codetitlewidth\arabic{code}}\ \rule{\eqboxwidth{codetitlewidth\arabic{code}}}{1pt}
                \item[codeupperwidth:] \eqboxwidth{codeupperwidth\arabic{code}}\ \rule{\eqboxwidth{codeupperwidth\arabic{code}}}{1pt}
                \item[codelowerwidth:] \eqboxwidth{codelowerewidth\arabic{code}}\ \rule{\eqboxwidth{codelowerwidth\arabic{code}}}{1pt}
              \end{enumerate}
            \item \emph{\textbackslash l\textunderscore\textunderscore listings\textunderscore autowidth\textunderscore \__listings_debug_argument_formatter:n {component}\textunderscore dim}
              \begin{enumerate}
                \item
                [
                  \underline
                  {
                    \__listings_debug_argument_formatter:n
                    {
                      component
                    }
                  }
                ] 
                \item[title:] \dim_use:N \l__listings_autowidth_title_dim \ \rule{\l__listings_autowidth_title_dim}{1pt}
                \item[upper:] \dim_use:N \l__listings_autowidth_upper_dim \ \rule{\l__listings_autowidth_upper_dim}{1pt}
                \item[lower:] \dim_use:N \l__listings_autowidth_lower_dim \ \rule{\l__listings_autowidth_lower_dim}{1pt}
              \end{enumerate}
        \end{enumerate}
        \rule{\textwidth}{1pt}
        }
  }
    \cs_new:Npn \__listings_debug_argument_formatter:n #1 {\textlangle #1 \textrangle}
  \bool_new:N \g__listings_autowidth_title_bool
  \bool_new:N \g__listings_autowidth_upper_bool
  \bool_new:N \g__listings_autowidth_lower_bool
  \bool_new:N \g__listings_debug_autowidth_bool
  % Values
  \tl_set:Nn \g__listings_linenos_tl {linenos}
  \tl_set:Nn \g__listings_breaklines_tl {breaklines}

\tcbset{
% expl3 equivalents
    % Help sourced from https://tex.stackexchange.com/a/748130/383565
  ,DimComparenNnT/.code~n~args={4}{\DimComparenNnT{#1}#2{#3}{\pgfkeysalso{#4}}}
  ,DimComparenNnTF/.code~n~args={5}{\DimComparenNnTF{#1}#2{#3}{\pgfkeysalso{#4}}{\pgfkeysalso{#5}}}
  ,BoolLazyOrnnT/.code~n~args={4}{\BoolLazyOrnnT{#1}{#2}{\pgfkeysalso{#3}}}
  ,BoolLazyOrnnTF/.code~n~args={4}{\BoolLazyOrnnTF{#1}{#2}{\pgfkeysalso{#3}}{\pgfkeysalso{#4}}}
  ,TLIfEqeeT/.code~n~args={3}{\TLIfEqeeT{#1}{#2}{\pgfkeysalso{#3}}}
  % ,TLIFEqeeT/.code~n~args={3}{\TLIfEqeeT{#1}{#2}{\pgfkeysalso{#3}}}
% Listing keys
  ,no~number/.code = {\tl_clear:N \g__listings_linenos_tl}
  ,autowidth@core/.code = 
  {
    % The line width \g__listings_autowidth@core_dim the box should be set to have (given this width is less than \textwidthoutside, i.e. doesn't overflow into the margins)
      \bool_if:NT \g__listings_autowidth_title_bool
      {
        \dim_set:Nn \l__listings_autowidth_title_dim 
        {
          \eqboxwidth{codetitlewidth\arabic{code}}
        }
      }
      \bool_if:NT \g__listings_autowidth_upper_bool
      {
        \dim_set:Nn \l__listings_autowidth_upper_dim
        {
          \eqboxwidth{codeupperwidth\arabic{code}}
        }
      }
      \bool_if:NT \g__listings_autowidth_lower_bool
      {
        \dim_set:Nn \l__listings_autowidth_lower_dim
        {
          \eqboxwidth{codelowerwidth\arabic{code}}
        }
      }
      \dim_set:Nn \g__listings_autowidth@core_dim 
      {
        \dim_max:nn 
        {\l__listings_autowidth_title_dim} 
        {
          \dim_max:nn
          {\l__listings_autowidth_upper_dim}
          {\l__listings_autowidth_lower_dim}
        }
      }
      \bool_if:nTF 
        {\g__listings_autowidth_upper_bool || \g__listings_autowidth_lower_bool} 
        {
          % Autowidth = upper or lower: Disable breaklines to allow \eqboxwidth{codeupperwidth\arabic{code}} to correctly measure the greatest line width. Enable breaklines if the line width overflows into the margins---autowidth ceases to work (the listing behaves as if autowidth wasn't used). (Why? I'm not sure; this is just the behavior I observed and wrote my code around.)
          \dim_compare:nNnTF
            % Relationship to compare 
            {\textwidthoutside} 
            < 
            {\__tcb_text_width_to_width:n {\g__listings_autowidth@core_dim}}
            % TF code execution
            {\tl_set:Nn \g__listings_breaklines_tl {breaklines}}
            {\tl_clear:N \g__listings_breaklines_tl }
        }
        {
          % Autowidth = title: the title might be smaller (in width) than the body (upper and lower parts). So, breaklines needs to be enabled to avoid overflowing contents. Why doesn't breaklines essentially disable autowidth in this case (unlike for autowidth = upper)? Breaklines is a minted key that affects only the minted environment---the upper part of the box. 
          \tl_set:Nn \g__listings_breaklines_tl {breaklines}
        }
    % autowidth functionality: setting the tcbkey 'text width'
      \dim_compare:nNnF
        % Relationship to compare
        {\textwidthoutside}
         <
        {\__tcb_text_width_to_width:n {\g__listings_autowidth@core_dim}}
        % F code execution
        {\pgfkeysalso{text~width = \g__listings_autowidth@core_dim}}
  }
  ,autowidth/.is~choice
    ,autowidth/title/.code = 
    {
      \bool_set_true:N \g__listings_autowidth_title_bool
      \pgfkeysalso{autowidth@core}
    } 
    ,autowidth/upper/.code = 
    {
      \bool_set_true:N \g__listings_autowidth_upper_bool
      \pgfkeysalso{autowidth@core}
    }
    ,autowidth/lower/.code = 
    {
      \bool_set_true:N \g__listings_autowidth_lower_bool
      \pgfkeysalso
      {
        ,listing~and~text
        ,autowidth@core
      }
    }
  ,debug/.is~choice
    ,debug/autowidth/.code = 
    {
      \bool_set_true:N \g__listings_debug_autowidth_bool
    } 
  ,Title/.style = 
  {
    title   = 
    {
      \sffamily\bfseries\color{\ColorCodeTitle}
        {
          \hspace{-\eqboxwidth{code\arabic{code}}+\titleleft}
          \__listings_autowidth_title:n {#1}
          \__listings_title_contents:n {#1}
        }
      }
  }
  ,OTitle/.code = {
    \tl_if_empty:nF {#1} {\tl_set:Nn \g__listings_OTitle_tl {~(#1)}}
  }
  ,vflush/.style = {before~skip = -\vlengthbeforelisting}
  ,vnormal/.style = {before~skip = \vlengthbetweencorrector}
% Base listing style
    % Help sourced from https://tex.stackexchange.com/a/748148/383565
    % codestylebase = { <Unparenthesized Title> }
  ,codestylebase/.style =
  {
    ,left                            = {\truelefthspace}
    ,listing~engine                  = minted
    ,minted~style                    = staroffice
    ,minted~options                  = 
      {
      \g__listings_breaklines_tl,
      autogobble,
      \g__listings_linenos_tl,
      numbersep = \numbersep
      }
    ,listing~only
    ,breakable
    ,enhanced
    ,skin                            = enhanced~jigsaw
    ,boxsep                          = \boxsep
    ,boxrule                         = \boxrule
    ,colback                         = \ColorCodeColback 
    ,Title                           = {#1}
    ,toptitle                        = 2mm
    ,bottomtitle                     = 2mm
    ,colbacktitle                    = \ColorCodeColbacktitle,
    ,borderline~west                 = {\BorderlineWestThickness}{0pt}{\ColorCodeBorderline}
    ,overlay~unbroken                = {
        \begin{tcbclipinterior}
            % Background for the code line numebrs
            \filldraw[\ColorCodeLineNumberBackground] 
            (frame.south~west) --
            (interior.north~west) {[rounded~corners=4] --
            ++(\eqboxwidth{code\arabic{code}}+\linenumberpadding,0) --
            ([xshift=\eqboxwidth{code\arabic{code}}+\linenumberpadding]frame.south~west)} --
            (frame.south~west);
            % Borderline west of \BorderlineWestThickness thickness
            \fill[\ColorCodeBorderline] (frame.south~west)
            rectangle ([xshift=\BorderlineWestThickness]frame.north~west);
        \end{tcbclipinterior}
    }
    ,overlay~first                   = {
        \begin{tcbclipinterior}
            % Background for the code line numebrs
            \filldraw[\ColorCodeLineNumberBackground] 
            (frame.south~west) --
            (interior.north~west) {[rounded~corners=4] --
            ++(\eqboxwidth{code\arabic{code}}+\linenumberpadding,0)} --
            ([xshift=\eqboxwidth{code\arabic{code}}+\linenumberpadding]frame.south~west) --
            (frame.south~west);
            % Borderline west of \BorderlineWestThickness thickness
            \fill[\ColorCodeBorderline] (frame.south~west)
            rectangle ([xshift=\BorderlineWestThickness]frame.north~west);
        \end{tcbclipinterior}
    }
    ,overlay~middle                  = {
        \begin{tcbclipinterior}
            % Background for the code line numebrs
            \fill[\ColorCodeLineNumberBackground] (frame.south~west)
            rectangle ([xshift=\eqboxwidth{code\arabic{code}}+\linenumberpadding]frame.north~west);
            % Borderline west of \BorderlineWestThickness thickness
            \fill[\ColorCodeBorderline] (frame.south~west)
            rectangle ([xshift=\BorderlineWestThickness]frame.north~west);
        \end{tcbclipinterior}
    }
    ,overlay~last                    = {
        \begin{tcbclipinterior}
            % Background for the code line numebrs
            \filldraw[\ColorCodeLineNumberBackground] 
            (frame.south~west) --
            (frame.north~west) --
            ++(\eqboxwidth{code\arabic{code}}+\linenumberpadding,0) {[rounded~corners=4] --
            ([xshift=\eqboxwidth{code\arabic{code}}+\linenumberpadding]frame.south~west)} --
            (frame.south~west);
            % Borderline west of \BorderlineWestThickness thickness
            \fill[\ColorCodeBorderline] (frame.south~west)
            rectangle ([xshift=\BorderlineWestThickness]frame.north~west);
        \end{tcbclipinterior}
    }
  }
% listing environment style
    % Help sourced from https://tex.stackexchange.com/a/748148/383565
    % codestlyenvironment = 
    % #1  { <displayed language name to be converted to minted language>}
    % #2  { <Unparentisized Title> }
  ,codestyleenvironment/.style~2~args =
  {
    ,code = {\DisplayedToPygments{#1}}
    ,minted~language                 = {\pygmentsname}
    ,codestylebase                   = {#2}
  }
% input listing style
    % Help sourced from https://tex.stackexchange.com/a/748148/383565
    % codestyleinput = 
    % #1  { <Listing file> }
    % #2  { <Unparentisized Title> }
  ,codestyleinput/.style~2~args =
  {
    ,listing~file   = {#1}
    ,code           = {\storefileextension{#1}
    \FileExtensionToLanguage{\fileextension}}
    ,minted~language = {\pygmentsname}
    ,codestylebase  = {#2}
  },
}
\ExplSyntaxOff
\makeatother

% Base listing style, continued
\renewcommand{\theFancyVerbLine}{
                \eqmakebox[code\arabic{code}][c]{
                    \textcolor{\ColorCodeLineNumber}{\scriptsize\arabic{FancyVerbLine}}
                }
            }
\renewcommand{\FancyVerbFormatLine}[1]{\eqsavebox{\codewidth}[codeupperwidth\arabic{code}]{#1}\fbox{#1}}

